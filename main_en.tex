%!BIB program = bibtex
% !TeX TS-program = xelatex

\documentclass{resume}
\ResumeName{Jiarui Ji}

% 如果想插入照片,请使用以下两个库.
% \usepackage{graphicx}
% \usepackage{tikz}

\begin{document}

\begin{textblock*}{2cm}(1cm, 1cm)
  {\includegraphics[height=1.3cm]{figure/ruc_uni_logo.png}}
  %{\includegraphics[height=1.3cm]{figure/Fudan_Uni_logo.svg.png}} % 你可以添加更多的学校 logo
\end{textblock*}

\ResumeContacts{
  \faPhone\ 18051801365,
  \ResumeUrl{mailto:2023100839@ruc.edu.cn}{\faEnvelope\ 2023100839@ruc.edu.cn},
  \ResumeUrl{https://github.com/2829397175}{\faGithub\ Ji-Cather}
}

\ResumeTitle

% \section{个人总结}
% \begin{itemize}
%   \item 
%   \item 积极与团队成员在开发过程中交互,有一定的团队开发经验,具备良好的沟通与展示能力.更多信息可参见 \ResumeUrl{https://github.com/2829397175}{GitHub}.
% \end{itemize}

\section{Education}
\ResumeItem
[Tongji University|B.E.]
{Tongji University}
[\textnormal{Data Science from the School of Electronic and Information Engineering|} B.E.]
[2019.09-2023.06]
\textbf{GPA: 4.86/5.00; Rank: 2/37}. \textbf{Outstanding Graduates from Shanghai} (Top 5\% in Shanghai)、\textbf{National Scholarship} (Top 0.2\% Nationwide)
\ResumeItem
[Renmin University of China|M.E.]
{Renmin University of China}
[\textnormal{Gaoling School of Artificial Intelligence|} M.E.]
[2023.09-Todate]
\textbf{English: CET-6 score 633; CET-4 score 626}

\section{Intern}

\ResumeItem{\textbf{Megvii}}
[Intern/C++]
[]
[2022.04-2023.06]
\begin{itemize}
  \item \textbf{Mge-Converter}: Developed opr converter from different deep-learning engines(e.g. pytorch) in Mge-Converter, \ResumeUrl{https://github.com/MegEngine/mgeconvert.git}{mgeconvert}.
\end{itemize}
\ResumeItem{\textbf{Alibaba Tongyi Lab}}
[Intern/Research]
[]
[2024.03-2025.03]
\begin{itemize}
  \item \textbf{GAG}: Developed GraphAgent-Generator (GAG), a parallelized social graph synthesis framework leveraging agent-item bipartite modeling. First-author paper has been accepted at ACL-2025. \ResumeUrl{http://arxiv.org/abs/2410.09824}{arXiv}. \ResumeUrl{https://github.com/Ji-Cather/GraphAgent}{Code}.
  \item \textbf{GAD}: Developed LLM-based predictive frameworks for dynamic graphs, addressing critical challenges in context-length constraints and domain variability. \ResumeUrl{https://arxiv.org/abs/2503.03258}{arXiv}.
  \item \textbf{GDGB}: Proposed GDGB, the first comprehensive benchmark for dynamic text-attributed graph (DyTAG) generation, featuring 8 text-rich graph datasets and a newly designed evaluation pipeline. \ResumeUrl{https://arxiv.org/pdf/2507.03267}{arXiv}. \ResumeUrl{https://gdgb-algo.github.io/}{Website}. \ResumeUrl{https://github.com/Lucas-PJ/GDGB-ALGO}{Code}.
\end{itemize}



\section{Research}
\ResumeItem{\textbf{SRAP-Agent}}
[]
[2023.07-2024.01]
\begin{itemize}
  \item Designed and implemented SRAP-Agent, a novel framework integrating LLMs into economic simulation for scarce public resource allocation (e.g., housing distribution in Beijing/Singapore). First-author paper accepted at EMNLP 2024. \ResumeUrl{http://arxiv.org/abs/2410.14152}{arXiv}. \ResumeUrl{https://github.com/Ji-Cather/SRAPAgent_Framework}{Code}
\end{itemize}

\ResumeItem{\textbf{CiteAgent}}
[]
[2024.03-2025.03]
\begin{itemize}
  \item Built an LLM-agent framework to model citation network evolution, revealing mechanisms behind power-law distributions, citation deflection, and diameter shrinkage in network science.
  \item Proposed dual simulation methodologies to systematically explain complex network dynamics, advancing understanding of emergent network properties.
  \item Code publicly available. \ResumeUrl{https://github.com/Ji-Cather/CiteAgent}{Code}.
\end{itemize}

\section[Skills]{Skills}
\begin{itemize}
  \item \textbf{Skills}: Git,GitHub,Markdown,Shell,SQL
  \item \textbf{Coding}: Python,C/C++,JavaScript, C\#
\end{itemize}

\end{document}
