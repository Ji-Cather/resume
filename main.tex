%!BIB program = bibtex
% !TeX TS-program = xelatex

\documentclass{resume}
\ResumeName{Jiarui Ji}

% 如果想插入照片,请使用以下两个库。
% \usepackage{graphicx}
% \usepackage{tikz}

\begin{document}

\begin{textblock*}{2cm}(1cm, 1cm)
  {\includegraphics[height=1.3cm]{figure/ruc_uni_logo.png}}
  %{\includegraphics[height=1.3cm]{figure/Fudan_Uni_logo.svg.png}} % 你可以添加更多的学校 logo
\end{textblock*}

\ResumeContacts{
  \faPhone\ 18051801365,
  \ResumeUrl{mailto:2023100839@ruc.edu.cn}{\faEnvelope\ 2023100839@ruc.edu.cn},
  \ResumeUrl{https://github.com/2829397175}{\faGithub\ Ji-Cather}
}

\ResumeTitle

% \section{个人总结}
% \begin{itemize}
%   \item 
%   \item 积极与团队成员在开发过程中交互,有一定的团队开发经验,具备良好的沟通与展示能力。更多信息可参见 \ResumeUrl{https://github.com/2829397175}{GitHub}。
% \end{itemize}

\section{教育经历}
\ResumeItem
[同济大学|本科生]
{同济大学}
[\textnormal{电子信息与工程学院 · 数据科学与大数据技术|} 工学学士]
[2019.09-2023.06]
\textbf{GPA: 4.86/5.00; Rank: 2/37}。曾获\textbf{上海市优秀毕业生} (上海市前 5\%)、\textbf{国家奖学金}(全国前 0.2\%)等荣誉。
\ResumeItem
[中国人民大学|硕士生]
{中国人民大学}
[\textnormal{高瓴人工智能学院 · 人工智能|} 工学硕士]
[2023.09-至今]
\textbf{英语水平: CET-6 633 分; CET-4 626 分}

\section{工作经历}

\ResumeItem{\textbf{Megvii-上海旷视科技有限公司}}
[实习生 / c++开发]
[]
[2022.04-2023.06]
\begin{itemize}
  \item \textbf{开发meg-converter}:开发了meg-converter中python侧的opr转换,代码库\ResumeUrl{https://github.com/MegEngine/mgeconvert.git}{mgeconvert}。
  \item \textbf{开发cuda算子}:开发了megengine框架下c++后端的group-norm算子,代码库\ResumeUrl{https://github.com/MegEngine/MegEngine}{MgeEngine}。
\end{itemize}


\section{项目经历}

\ResumeItem{\textbf{千针万确——自动化静脉采血机器人}}
[\ 项目成员]
[]
[2021.03-2022.03]
\begin{itemize}
  \item \textbf{大学生国家级创新项目优秀结题},\textbf{上海市挑战杯一等奖}
  \item 该智能静脉穿刺采血机器人是一款基于深度学习的多模态图像引导静脉穿刺机器人系统,围绕近红外
  和超声图像对于手背表层血管进行识别,控制机器进行穿刺。
  \item 一项国家发明专利。
\end{itemize}

\ResumeItem{\textbf{SRAP-Agent}}
[\ 项目负责人]
[]
[2023.07-2024.01]
\begin{itemize}
  \item 使用LLM-agent搭建公共稀缺资源配置的框架,实验模拟北京市、新加坡等的房屋分配过程。
  \item 一作论文被录用为EMNLP24.findings - SRAP-Agent: Simulating and Optimizing Scarce Resource Allocation Policy with LLM-based Agent.  
  \item 代码已开源 \ResumeUrl{https://github.com/Ji-Cather/SRAPAgent_Framework}{SRAP-Agent}
  \item 在这项工作中,我们提出了SRAP-Agent框架,将LLM集成到经济模拟中,旨在弥合理论模型和现实经济场景之间的差距。以公共住房分配场景为例,我们进行了广泛的政策模拟实验,以验证SRAP-Agent的可行性和有效性,并使用了政策优化算法POA对于指定政策目标进行优化。
\end{itemize}

\ResumeItem{\textbf{GraphAgent-Generator}}
[\ 项目负责人]
[]
[2024.03-2025.03]
\begin{itemize}
  \item 使用LLM-agent搭建graph-generator,基于智能体模拟生成多样化的社会网络
  \item 代码已开源\ResumeUrl{https://github.com/Ji-Cather/GraphAgent}{GraphAgent}
\end{itemize}

\ResumeItem{\textbf{CiteAgent}}
[\ 项目负责人]
[]
[2024.03-2025.03]
\begin{itemize}
  \item 使用LLM-agent模拟引文网络生成过程,探究网络科学各项发现的背后机理
  \item 我们提出了两种模拟实验的方法,对网络幂律分布、引文扭曲现象、直径缩减现象提出了合理的机理解释
  \item 代码已开源\ResumeUrl{https://github.com/Ji-Cather/CiteAgent}{CiteAgent}
\end{itemize}

\section[技术及其他]{技术及其他}
\begin{itemize}
  \item \textbf{技能}: Git,GitHub,Markdown,Shell,SQL
  \item \textbf{语言}: Python,C/C++,JavaScript, C#
\end{itemize}

\end{document}
